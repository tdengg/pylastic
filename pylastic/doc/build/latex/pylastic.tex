% Generated by Sphinx.
\def\sphinxdocclass{report}
\documentclass[letterpaper,10pt,english]{sphinxmanual}
\usepackage[utf8]{inputenc}
\DeclareUnicodeCharacter{00A0}{\nobreakspace}
\usepackage[T1]{fontenc}
\usepackage{babel}
\usepackage{times}
\usepackage[Bjarne]{fncychap}
\usepackage{longtable}
\usepackage{sphinx}
\usepackage{multirow}


\title{pylastic Documentation}
\date{June 10, 2014}
\release{0.1}
\author{Dengg Thomas}
\newcommand{\sphinxlogo}{}
\renewcommand{\releasename}{Release}
\makeindex

\makeatletter
\def\PYG@reset{\let\PYG@it=\relax \let\PYG@bf=\relax%
    \let\PYG@ul=\relax \let\PYG@tc=\relax%
    \let\PYG@bc=\relax \let\PYG@ff=\relax}
\def\PYG@tok#1{\csname PYG@tok@#1\endcsname}
\def\PYG@toks#1+{\ifx\relax#1\empty\else%
    \PYG@tok{#1}\expandafter\PYG@toks\fi}
\def\PYG@do#1{\PYG@bc{\PYG@tc{\PYG@ul{%
    \PYG@it{\PYG@bf{\PYG@ff{#1}}}}}}}
\def\PYG#1#2{\PYG@reset\PYG@toks#1+\relax+\PYG@do{#2}}

\def\PYG@tok@gd{\def\PYG@tc##1{\textcolor[rgb]{0.63,0.00,0.00}{##1}}}
\def\PYG@tok@gu{\let\PYG@bf=\textbf\def\PYG@tc##1{\textcolor[rgb]{0.50,0.00,0.50}{##1}}}
\def\PYG@tok@gt{\def\PYG@tc##1{\textcolor[rgb]{0.00,0.25,0.82}{##1}}}
\def\PYG@tok@gs{\let\PYG@bf=\textbf}
\def\PYG@tok@gr{\def\PYG@tc##1{\textcolor[rgb]{1.00,0.00,0.00}{##1}}}
\def\PYG@tok@cm{\let\PYG@it=\textit\def\PYG@tc##1{\textcolor[rgb]{0.25,0.50,0.56}{##1}}}
\def\PYG@tok@vg{\def\PYG@tc##1{\textcolor[rgb]{0.73,0.38,0.84}{##1}}}
\def\PYG@tok@m{\def\PYG@tc##1{\textcolor[rgb]{0.13,0.50,0.31}{##1}}}
\def\PYG@tok@mh{\def\PYG@tc##1{\textcolor[rgb]{0.13,0.50,0.31}{##1}}}
\def\PYG@tok@cs{\def\PYG@tc##1{\textcolor[rgb]{0.25,0.50,0.56}{##1}}\def\PYG@bc##1{\colorbox[rgb]{1.00,0.94,0.94}{##1}}}
\def\PYG@tok@ge{\let\PYG@it=\textit}
\def\PYG@tok@vc{\def\PYG@tc##1{\textcolor[rgb]{0.73,0.38,0.84}{##1}}}
\def\PYG@tok@il{\def\PYG@tc##1{\textcolor[rgb]{0.13,0.50,0.31}{##1}}}
\def\PYG@tok@go{\def\PYG@tc##1{\textcolor[rgb]{0.19,0.19,0.19}{##1}}}
\def\PYG@tok@cp{\def\PYG@tc##1{\textcolor[rgb]{0.00,0.44,0.13}{##1}}}
\def\PYG@tok@gi{\def\PYG@tc##1{\textcolor[rgb]{0.00,0.63,0.00}{##1}}}
\def\PYG@tok@gh{\let\PYG@bf=\textbf\def\PYG@tc##1{\textcolor[rgb]{0.00,0.00,0.50}{##1}}}
\def\PYG@tok@ni{\let\PYG@bf=\textbf\def\PYG@tc##1{\textcolor[rgb]{0.84,0.33,0.22}{##1}}}
\def\PYG@tok@nl{\let\PYG@bf=\textbf\def\PYG@tc##1{\textcolor[rgb]{0.00,0.13,0.44}{##1}}}
\def\PYG@tok@nn{\let\PYG@bf=\textbf\def\PYG@tc##1{\textcolor[rgb]{0.05,0.52,0.71}{##1}}}
\def\PYG@tok@no{\def\PYG@tc##1{\textcolor[rgb]{0.38,0.68,0.84}{##1}}}
\def\PYG@tok@na{\def\PYG@tc##1{\textcolor[rgb]{0.25,0.44,0.63}{##1}}}
\def\PYG@tok@nb{\def\PYG@tc##1{\textcolor[rgb]{0.00,0.44,0.13}{##1}}}
\def\PYG@tok@nc{\let\PYG@bf=\textbf\def\PYG@tc##1{\textcolor[rgb]{0.05,0.52,0.71}{##1}}}
\def\PYG@tok@nd{\let\PYG@bf=\textbf\def\PYG@tc##1{\textcolor[rgb]{0.33,0.33,0.33}{##1}}}
\def\PYG@tok@ne{\def\PYG@tc##1{\textcolor[rgb]{0.00,0.44,0.13}{##1}}}
\def\PYG@tok@nf{\def\PYG@tc##1{\textcolor[rgb]{0.02,0.16,0.49}{##1}}}
\def\PYG@tok@si{\let\PYG@it=\textit\def\PYG@tc##1{\textcolor[rgb]{0.44,0.63,0.82}{##1}}}
\def\PYG@tok@s2{\def\PYG@tc##1{\textcolor[rgb]{0.25,0.44,0.63}{##1}}}
\def\PYG@tok@vi{\def\PYG@tc##1{\textcolor[rgb]{0.73,0.38,0.84}{##1}}}
\def\PYG@tok@nt{\let\PYG@bf=\textbf\def\PYG@tc##1{\textcolor[rgb]{0.02,0.16,0.45}{##1}}}
\def\PYG@tok@nv{\def\PYG@tc##1{\textcolor[rgb]{0.73,0.38,0.84}{##1}}}
\def\PYG@tok@s1{\def\PYG@tc##1{\textcolor[rgb]{0.25,0.44,0.63}{##1}}}
\def\PYG@tok@gp{\let\PYG@bf=\textbf\def\PYG@tc##1{\textcolor[rgb]{0.78,0.36,0.04}{##1}}}
\def\PYG@tok@sh{\def\PYG@tc##1{\textcolor[rgb]{0.25,0.44,0.63}{##1}}}
\def\PYG@tok@ow{\let\PYG@bf=\textbf\def\PYG@tc##1{\textcolor[rgb]{0.00,0.44,0.13}{##1}}}
\def\PYG@tok@sx{\def\PYG@tc##1{\textcolor[rgb]{0.78,0.36,0.04}{##1}}}
\def\PYG@tok@bp{\def\PYG@tc##1{\textcolor[rgb]{0.00,0.44,0.13}{##1}}}
\def\PYG@tok@c1{\let\PYG@it=\textit\def\PYG@tc##1{\textcolor[rgb]{0.25,0.50,0.56}{##1}}}
\def\PYG@tok@kc{\let\PYG@bf=\textbf\def\PYG@tc##1{\textcolor[rgb]{0.00,0.44,0.13}{##1}}}
\def\PYG@tok@c{\let\PYG@it=\textit\def\PYG@tc##1{\textcolor[rgb]{0.25,0.50,0.56}{##1}}}
\def\PYG@tok@mf{\def\PYG@tc##1{\textcolor[rgb]{0.13,0.50,0.31}{##1}}}
\def\PYG@tok@err{\def\PYG@bc##1{\fcolorbox[rgb]{1.00,0.00,0.00}{1,1,1}{##1}}}
\def\PYG@tok@kd{\let\PYG@bf=\textbf\def\PYG@tc##1{\textcolor[rgb]{0.00,0.44,0.13}{##1}}}
\def\PYG@tok@ss{\def\PYG@tc##1{\textcolor[rgb]{0.32,0.47,0.09}{##1}}}
\def\PYG@tok@sr{\def\PYG@tc##1{\textcolor[rgb]{0.14,0.33,0.53}{##1}}}
\def\PYG@tok@mo{\def\PYG@tc##1{\textcolor[rgb]{0.13,0.50,0.31}{##1}}}
\def\PYG@tok@mi{\def\PYG@tc##1{\textcolor[rgb]{0.13,0.50,0.31}{##1}}}
\def\PYG@tok@kn{\let\PYG@bf=\textbf\def\PYG@tc##1{\textcolor[rgb]{0.00,0.44,0.13}{##1}}}
\def\PYG@tok@o{\def\PYG@tc##1{\textcolor[rgb]{0.40,0.40,0.40}{##1}}}
\def\PYG@tok@kr{\let\PYG@bf=\textbf\def\PYG@tc##1{\textcolor[rgb]{0.00,0.44,0.13}{##1}}}
\def\PYG@tok@s{\def\PYG@tc##1{\textcolor[rgb]{0.25,0.44,0.63}{##1}}}
\def\PYG@tok@kp{\def\PYG@tc##1{\textcolor[rgb]{0.00,0.44,0.13}{##1}}}
\def\PYG@tok@w{\def\PYG@tc##1{\textcolor[rgb]{0.73,0.73,0.73}{##1}}}
\def\PYG@tok@kt{\def\PYG@tc##1{\textcolor[rgb]{0.56,0.13,0.00}{##1}}}
\def\PYG@tok@sc{\def\PYG@tc##1{\textcolor[rgb]{0.25,0.44,0.63}{##1}}}
\def\PYG@tok@sb{\def\PYG@tc##1{\textcolor[rgb]{0.25,0.44,0.63}{##1}}}
\def\PYG@tok@k{\let\PYG@bf=\textbf\def\PYG@tc##1{\textcolor[rgb]{0.00,0.44,0.13}{##1}}}
\def\PYG@tok@se{\let\PYG@bf=\textbf\def\PYG@tc##1{\textcolor[rgb]{0.25,0.44,0.63}{##1}}}
\def\PYG@tok@sd{\let\PYG@it=\textit\def\PYG@tc##1{\textcolor[rgb]{0.25,0.44,0.63}{##1}}}

\def\PYGZbs{\char`\\}
\def\PYGZus{\char`\_}
\def\PYGZob{\char`\{}
\def\PYGZcb{\char`\}}
\def\PYGZca{\char`\^}
\def\PYGZsh{\char`\#}
\def\PYGZpc{\char`\%}
\def\PYGZdl{\char`\$}
\def\PYGZti{\char`\~}
% for compatibility with earlier versions
\def\PYGZat{@}
\def\PYGZlb{[}
\def\PYGZrb{]}
\makeatother

\begin{document}

\maketitle
\tableofcontents
\phantomsection\label{index::doc}


Contents:
\phantomsection\label{index:module-distort}\index{distort (module)}
Get DEFORMATION MATRIX for given spacegroup number and Lagranian strain.
\begin{description}
\item[{Methods:}] \leavevmode
get\_eta        : returns Lagrangian strain

get\_sgn        : returns spacegroup number

get\_strainType : returns current strain (deformation) type

get\_defMatrix  : returns deformation matrix

get\_strainList : returns list of deformation types for specific crystal symmetry

get\_V0

set\_eta        : set Lagrangian strain

set\_sgn        : set spacegroup number

set\_strainType : set current strain(deformation) type

set\_defMatrix  : calculates deformation matrix

set\_strainList : finds list of deformation types for specific crystal symmetry

set\_V0

\end{description}

Example:

\begin{Verbatim}[commandchars=\\\{\}]
\PYG{k+kn}{import} \PYG{n+nn}{distort}

\PYG{n}{dist} \PYG{o}{=} \PYG{n}{distort}\PYG{o}{.}\PYG{n}{Distort}\PYG{p}{(}\PYG{p}{)}                 \PYG{c}{\PYGZsh{} generate instance of distortion object}
\PYG{n}{dist}\PYG{o}{.}\PYG{n}{sgn} \PYG{o}{=} \PYG{n}{sgn}                           \PYG{c}{\PYGZsh{} set spacegroup number (int)}
\PYG{n}{dist}\PYG{o}{.}\PYG{n}{set\PYGZus{}strainList}\PYG{p}{(}\PYG{p}{)}                    \PYG{c}{\PYGZsh{} set strain list according to space group number}

\PYG{n}{strainType} \PYG{o}{=} \PYG{n+nb}{next}\PYG{p}{(}\PYG{n}{dist}\PYG{o}{.}\PYG{n}{strainList\PYGZus{}iter}\PYG{p}{)}  \PYG{c}{\PYGZsh{} when using the iterator property --\textgreater{} get first distortion type}
\PYG{n}{dist}\PYG{o}{.}\PYG{n}{eta} \PYG{o}{=} \PYG{n}{eta}                           \PYG{c}{\PYGZsh{} define lagrangian strain (float)}
\PYG{n}{dist}\PYG{o}{.}\PYG{n}{set\PYGZus{}strainType}\PYG{p}{(}\PYG{n}{strainType}\PYG{p}{)}          \PYG{c}{\PYGZsh{} set strain type (string)}
\PYG{n}{dist}\PYG{o}{.}\PYG{n}{set\PYGZus{}defMatrix}\PYG{p}{(}\PYG{p}{)}                     \PYG{c}{\PYGZsh{} ste deformation matrix}
\PYG{n}{dist}\PYG{o}{.}\PYG{n}{get\PYGZus{}defMatrix}\PYG{p}{(}\PYG{p}{)}                     \PYG{c}{\PYGZsh{} get deformation matrix}
\end{Verbatim}
\index{Distort (class in distort)}

\begin{fulllineitems}
\phantomsection\label{index:distort.Distort}\pysiglinewithargsret{\strong{class }\code{distort.}\bfcode{Distort}}{\emph{volumeconserving=False}, \emph{mthd='Energy'}, \emph{order=2}}{}
Generate DEFORMATION MATRIX for given spacegroup number and Lagranian strain.
arguments:
\begin{quote}

volumecoserving: (True/False)
mthd: (`Energy'/'Stress') method of calculation
order: (2/3) specify order of elastic constants
\end{quote}

\end{fulllineitems}

\phantomsection\label{index:module-elatoms}\index{elatoms (module)}
Created on May 23, 2014

@author: t.dengg
\index{ElAtoms (class in elatoms)}

\begin{fulllineitems}
\phantomsection\label{index:elatoms.ElAtoms}\pysigline{\strong{class }\code{elatoms.}\bfcode{ElAtoms}}~\begin{quote}

classdocs:
ASE Atoms like object.
\begin{description}
\item[{Methods:}] \leavevmode
set\_cell
set\_natom
set\_scale
set\_species
set\_workdir
set\_poscar
set\_poscarnew
distort
poscarToAtoms
atomsToPoscar

get\_cell
get\_natom
get\_scale
get\_species
get\_workdir
get\_poscar
get\_poscarnew

\end{description}
\end{quote}

Example:

\begin{Verbatim}[commandchars=\\\{\}]
\PYG{n}{poscar} \PYG{o}{=} \PYG{n}{POS}\PYG{p}{(}\PYG{l+s}{'}\PYG{l+s}{POSCAR}\PYG{l+s}{'}\PYG{p}{)}\PYG{o}{.}\PYG{n}{read\PYGZus{}pos}\PYG{p}{(}\PYG{p}{)}
\PYG{n}{structures} \PYG{o}{=} \PYG{n}{Structures}\PYG{p}{(}\PYG{p}{)}
\end{Verbatim}

Generate distortion:

\begin{Verbatim}[commandchars=\\\{\}]
\PYG{n}{atom1} \PYG{o}{=} \PYG{n}{ElAtoms}\PYG{p}{(}\PYG{p}{)}
\PYG{n}{atom1}\PYG{o}{.}\PYG{n}{poscarToAtoms}\PYG{p}{(}\PYG{n}{poscar}\PYG{p}{)}
\PYG{n}{atom1}\PYG{o}{.}\PYG{n}{distort}\PYG{p}{(}\PYG{n}{eta}\PYG{o}{=}\PYG{l+m+mf}{0.01}\PYG{p}{,} \PYG{n}{strainType\PYGZus{}index} \PYG{o}{=} \PYG{l+m+mi}{0}\PYG{p}{)}
\PYG{n}{structures}\PYG{o}{.}\PYG{n}{append\PYGZus{}structure}\PYG{p}{(}\PYG{n}{atom1}\PYG{p}{)}
\end{Verbatim}

Generate another distortion:

\begin{Verbatim}[commandchars=\\\{\}]
\PYG{n}{atom2} \PYG{o}{=} \PYG{n}{ElAtoms}\PYG{p}{(}\PYG{p}{)}
\PYG{n}{atom2}\PYG{o}{.}\PYG{n}{poscarToAtoms}\PYG{p}{(}\PYG{n}{poscar}\PYG{p}{)}
\PYG{n}{atom2}\PYG{o}{.}\PYG{n}{distort}\PYG{p}{(}\PYG{n}{eta}\PYG{o}{=}\PYG{l+m+mf}{0.01}\PYG{p}{,} \PYG{n}{strainType\PYGZus{}index} \PYG{o}{=} \PYG{l+m+mi}{1}\PYG{p}{)}
\PYG{n}{structures}\PYG{o}{.}\PYG{n}{append\PYGZus{}structure}\PYG{p}{(}\PYG{n}{atom2}\PYG{p}{)}

\PYG{n}{structures}\PYG{o}{.}\PYG{n}{write\PYGZus{}structures}\PYG{p}{(}\PYG{p}{)}
\end{Verbatim}

\end{fulllineitems}

\index{Structures (class in elatoms)}

\begin{fulllineitems}
\phantomsection\label{index:elatoms.Structures}\pysigline{\strong{class }\code{elatoms.}\bfcode{Structures}}
Generate a series of distorted structures.
\begin{description}
\item[{Methods:}] \leavevmode
set\_fname
write\_structures
append\_structure
get\_structures

\end{description}

\end{fulllineitems}

\phantomsection\label{index:module-vaspIO}\index{vaspIO (module)}\phantomsection\label{index:module-postprocess}\index{postprocess (module)}\phantomsection\label{index:module-analyze}\index{analyze (module)}\index{Energy (class in analyze)}

\begin{fulllineitems}
\phantomsection\label{index:analyze.Energy}\pysiglinewithargsret{\strong{class }\code{analyze.}\bfcode{Energy}}{\emph{strain}, \emph{energy}, \emph{V0}}{}
\end{fulllineitems}



\chapter{Indices and tables}
\label{index:indices-and-tables}\label{index:welcome-to-pylastic-s-documentation}\begin{itemize}
\item {} 
\emph{genindex}

\item {} 
\emph{modindex}

\item {} 
\emph{search}

\end{itemize}


\renewcommand{\indexname}{Python Module Index}
\begin{theindex}
\def\bigletter#1{{\Large\sffamily#1}\nopagebreak\vspace{1mm}}
\bigletter{a}
\item {\texttt{analyze}}, \pageref{index:module-analyze}
\indexspace
\bigletter{d}
\item {\texttt{distort}}, \pageref{index:module-distort}
\indexspace
\bigletter{e}
\item {\texttt{elatoms}}, \pageref{index:module-elatoms}
\indexspace
\bigletter{p}
\item {\texttt{postprocess}}, \pageref{index:module-postprocess}
\indexspace
\bigletter{v}
\item {\texttt{vaspIO}}, \pageref{index:module-vaspIO}
\end{theindex}

\renewcommand{\indexname}{Index}
\printindex
\end{document}
